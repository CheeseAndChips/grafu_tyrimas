\documentclass[bibliography=totoc]{article}
\usepackage{titling}
\usepackage{listings}
%\usepackage[nottoc,numbib]{tocbibind}
\usepackage[nottoc]{tocbibind}

%\documentclass[bibliography=totoc]{scrartcl}

\usepackage[lithuanian]{babel}
\usepackage[utf8]{inputenc}
\def\LTfontencoding{L7x}
%\PrerenderUnicode{ąčęėįšųūž}
\usepackage[\LTfontencoding]{fontenc}
\usepackage[T1]{fontenc}
\providecommand{\mathdefault}[1]{#1}


\usepackage{geometry}
 \geometry{
 a4paper,
 total={170mm,257mm},
 left=20mm,
 top=20mm,
 }

\usepackage{natbib}
\usepackage{amssymb}
\usepackage{amsmath}
\usepackage{multicol}
\usepackage{tikz}
\usetikzlibrary{arrows, automata}
\usepackage{algorithm}
\usepackage{algorithmic}
% \usepackage{algpseudocode}
\usepackage{wrapfig}
\usepackage{graphicx}
\usepackage{float}
\usepackage{caption}
\usepackage{subcaption}
\usepackage{hyperref}
\newtheorem{definition}{Apibrėžimas}
\newtheorem{example}{Pavyzdys}
\newtheorem{experiment}{Grafas}

\begin{document}

\begin{titlepage}
    \begin{center}
        \vspace*{1cm}
 
        \Huge
        \textbf{Vilniaus Universitetas}\\
         \textbf{Informatikos institutas}\\
        
        \vspace{6.5cm}
        \textbf{Joris Pevcevičius}\\
       \textit{ informatikos specialybė, \\
                 III kursas, 1 grupė\\}
        
        \textbf{Ausų algoritmo taikymas}\\
          
                 
 
        \vfill
        Vilnius -- 2025
    \end{center}
\end{titlepage}
\tableofcontents
\newpage
\section*{Įvadas}
\label{sec:ivadas}
\addcontentsline{toc}{section}{\nameref{sec:ivadas}}
Šiame darbe tiriama grafų $G_{n,p}$ klasė naudojant chain decomposition algoritmą \citep{schmidt2013simple} iš NetworkX bibliotekos. Rezultatai atvaizduojami matplotlib bibliotekos pagalba.

\newpage
\section{Apibrėžimai. Tiriamoji grafų klasė}

Kadangi ne visi sugeneruojami grafai bus jungūs, reikės ieškoti didžiausios jungios komponentės.
\begin{definition}[Jungi komponentė]\citep{Graf}
Grafo $G=(V,E)$ jungia komponente vadinsime grafo $G$ porgrafį $H\subseteq G$ tokį, kad $H$ yra jungus grafas, t.y.
$H=(V_H, E_H): \forall v,u\in V, \exists P(v,u), v\neq u$, kur $P(v,u)$ yra kelias iš viršūnės $v$ į viršūnę $u$.
\end{definition}

\begin{definition}[$G(n,p)$ atsitiktinis grafas]
Tegu $n \in \mathbb{N}$ ir $p \in [0,1]$. $G(n, p)$ yra tikimybinis modelis neorientuotam grafui su $n$ viršūnių, kur kiekviena skirtingų viršūnių pora $(v_i, v_j)$ yra sujungiama briauna su nepriklausoma tikimybe $p$.
\end{definition}


\newpage
\section{Algoritmai}

DFS algoritmas, kuris bus naudojamas
\begin{algorithm}[h]\citep{CormenIntoAlg}
    \caption{DFS(G,cc,v,visited)}
    \begin{algorithmic}[1]
        \STATE $visited[v] := True$
        \STATE $cc\leftarrow v$
        \FORALL{$u\in ADJ(v)$}
            \IF{$visited[u]=False$}
                \STATE $cc := DFS(G,cc,u,visited)$
            \ENDIF
        \ENDFOR
        \RETURN $cc$
    \end{algorithmic}
\end{algorithm}    

\begin{algorithm}[h]\citep{CormenIntoAlg}
    \caption{CCs(G)}
    \begin{algorithmic}[1]
        \STATE $ccs := []$
        \STATE $visited := []$
        \FORALL{$v\in G$}
            \STATE $visited \leftarrow False$
        \ENDFOR
        \FORALL{$v\in G$}
            \IF{$visited[v]=False$}
                \STATE $cc := []$
                \STATE $ccs \leftarrow DFS(G,cc,v,visited)$
            \ENDIF
        \ENDFOR
        \STATE $CCS := []$
        \FORALL{$c \in ccs$}
            \STATE $CCS\leftarrow Graph(G,ccs)$
        \ENDFOR
        \RETURN $ccs$
    \end{algorithmic}
\end{algorithm}

\begin{algorithm}[h]\citep{schmidt2013simple}
    \caption{Check}
    \textbf{Duota:} Grafas $G=(V,E)$\\
    \textbf{Rasti:} Ar grafas 2-connected, 2-edge-connected, ar nei toks, nei toks.

    \begin{algorithmic}[1]
        \STATE Compute a DFS-tree $T$ of $G$
        \STATE Compute a chain decomposition $C$; mark every visited edge
        \IF{$G$ contains an unvisited edge} 
            \STATE output "\texttt{not 2-edge-connected}"
        \ELSIF{there is a cycle in $C$ different from $C_1$}
            \STATE output "\texttt{2-edge-connected but not 2-connected}"
        \ELSE
            \STATE output "\texttt{2-connected}"
        \ENDIF
    \end{algorithmic}
\end{algorithm}

\clearpage
\newpage
\section{Algoritmų realizacija ir tyrimų eiga}
Algoritmo realizacijai naudosime NetworkX bibliotekos implementacija skirta Python kalbai. Konkrečiai bus naudojama \texttt{networkx.chain\_decomposition} funkcija, kuri atiduoda grandinių iteratorių. Grandinė šiuo atveju bus briaunų sąrašas.

Generuosime atsitiktinius $G_{n,p}$ grafus. Pasirinksime aibes $N$ ir $P$, kuriose bus duotos bandomos $n$ ir $p$ reikšmės ir generuosime po $k$ grafų kiekvienai $N \times P$ reikšmei. Kiekvienam grafui $G = (V, E)$ surasime didžiausią jungią komponentę $G' = (V', E')$ ir naudosime chain decomposition algoritmą. Užsaugosime kiek tiltų atrandama grafe $G'$ ir ar grafas $G'$ yra 2-jungus. Taip pat užsaugome pradinio grafo dydį: $|V|, |E|$.

Šiam tyrimui parinksime $|N| = |P| = 100$ ir $k = 1000$. Mažiausia $p$ reikšmė - $10^{-3}$, didžiausia - $0.5$. Mažiausia $n$ reikšmė - $31$, didžiausia - $1000$.

\newpage
\section{Tyrimų rezultatai ir jų aptarimas}
Kadangi analizuojame didžiausią jungią komponentę, pavaizduojame kokia dalis grafo višūnių pateks į galutinę komponentę ir surasime šių reikšmių aritmetinį vidurkį visiems $k$ grafų. Analogiškai pavaizduojame kokia dalis sugeneruotų komponenčių yra 2-jungios. Komponentės dydį lyginame su funkcija $p = \frac{\ln n}{n}$, nes tai yra $G_{n,p}$ grafų jungumo riba. \citep{erd6s1960evolution}

\begin{figure}[H]
    \begin{subfigure}{.5\textwidth}
        \begin{center}
            \includegraphics{./plots/plot1.pdf}
        \end{center}
        \caption{grafo komponentės dydis ir funkcija $p = \frac{\ln n}{n}$}
    \end{subfigure}
    \begin{subfigure}{.5\textwidth}
        \begin{center}
            \includegraphics{./plots/plot2.pdf}
        \end{center}
        \caption{dalis grafų, kurie yra 2-jungūs}
    \end{subfigure}
    \caption{}
\end{figure}

Pavaizduojame kokia dalis grafo komponentės briaunų yra tiltai. Kadangi prie pereinamosios ribos tikėtina, kad bus pastebima įvairesni grafai, taip pat pavaizduojame šios dalies standartinį nuokrypį.
\begin{figure}[H]
    \begin{subfigure}{.5\textwidth}
        \begin{center}
            \includegraphics{./plots/plot3.pdf}
        \end{center}
        \caption{kokia dalis briaunų yra tiltai}
    \end{subfigure}
    \begin{subfigure}{.5\textwidth}
        \begin{center}
            \includegraphics{./plots/plot4.pdf}
        \end{center}
        \caption{tiltų dalies standartinis nuokrypis}
    \end{subfigure}
    \caption{}
\end{figure}

\newpage 
\section*{Išvados}
\label{sec:isvados}
\addcontentsline{toc}{section}{\nameref{sec:isvados}}
\begin{itemize}
    \item{Pastebėta pereinamoji juosta, kurioje komponentės tampa mažiau jungios. Nepanašu, kad juosta sektų $p = \frac{\ln n}{n}$.}
    \item{Toje pačioje juostoje pastebimas didelis tiltų dalies standartinis nuokrypis.}
\end{itemize}

\newpage 
\bibliographystyle{plain}
\bibliography{reference}
\newpage 
\appendix
\section*{Priedai. Kodas}
Kodas pasiekiamas \url{https://github.com/CheeseAndChips/grafu\_tyrimas.git}

Rezultatų failai:
\begin{itemize}
    \item{\url{http://karklas.mif.vu.lt/~jope9155/grafu\_tyrimas/out\_large.gz}}
    \item{\url{http://karklas.mif.vu.lt/~jope9155/grafu\_tyrimas/out\_small.gz}}
\end{itemize}
\label{sec:priedas}
\addcontentsline{toc}{section}{\nameref{sec:priedas}}
\definecolor{codegreen}{rgb}{0,0.6,0}
\definecolor{codegray}{rgb}{0.5,0.5,0.5}
\definecolor{codepurple}{rgb}{0.58,0,0.82}
\definecolor{backcolour}{rgb}{1,1,1}
\lstdefinestyle{mystyle}{
    backgroundcolor=\color{backcolour},   
    commentstyle=\color{codegreen},
    keywordstyle=\color{magenta},
    numberstyle=\tiny\color{codegray},
    stringstyle=\color{codepurple},
    basicstyle=\footnotesize,
    breakatwhitespace=false,         
    breaklines=true,                 
    captionpos=b,                    
    keepspaces=true,                 
    numbers=left,                    
    numbersep=5pt,                  
    showspaces=false,                
    showstringspaces=false,
    showtabs=false,                  
    tabsize=2
}
 
\lstset{style=mystyle}
main.py
\lstinputlisting[language=Python]{../py/main.py}
show.py
\lstinputlisting[language=Python]{../py/show.py}


\end{document}


\documentclass[bibliography=totoc]{article}
\usepackage{titling}
\usepackage{listings}
%\usepackage[nottoc,numbib]{tocbibind}
\usepackage[nottoc]{tocbibind}

%\documentclass[bibliography=totoc]{scrartcl}

\usepackage[lithuanian]{babel}
\usepackage[utf8]{inputenc}
\def\LTfontencoding{L7x}
%\PrerenderUnicode{ąčęėįšųūž}
\usepackage[\LTfontencoding]{fontenc}
\usepackage[T1]{fontenc}


\usepackage{geometry}
 \geometry{
 a4paper,
 total={170mm,257mm},
 left=20mm,
 top=20mm,
 }

\usepackage{natbib}
\usepackage{amssymb}
\usepackage{amsmath}
\usepackage{multicol}
\usepackage{tikz}
\usetikzlibrary{arrows, automata}
\usepackage{algorithm}
\usepackage{algorithmic}
\usepackage{wrapfig}
\usepackage{graphicx}
\usepackage{float}
\usepackage{caption}
\usepackage{subcaption}
\usepackage{hyperref}
\newtheorem{definition}{Apibrėžimas}
\newtheorem{example}{Pavyzdys}
\newtheorem{experiment}{Grafas}

\begin{document}

\begin{titlepage}
    \begin{center}
        \vspace*{1cm}
 
        \Huge
        \textbf{Vilniaus Universitetas}\\
         \textbf{Informatikos institutas}\\
        
        \vspace{6.5cm}
        \textbf{Vardenis Pavardenis}\\
       \textit{ informatikos specialybė, \\
                 II kursas, NN grupė\\}
        
        \textbf{Sunkiausios komponentės paieška}\\
          
                 
 
        \vfill
        Vilnius -- 2023
    \end{center}
\end{titlepage}
\tableofcontents
\newpage
\section*{Įvadas}
\label{sec:ivadas}
\addcontentsline{toc}{section}{\nameref{sec:ivadas}}
Šiame darbe tiriama grafų GGGGGG klasė naudojant  NNNNNNN algoritmą .....
<aprašoma glaustai grafų klasė, tirimo tikslas, etapai, instrumentai ....  >

\newpage
\section{Apibrėžimai. Tiriamoji grafų klasė}
<Galima kas tinka permesti iš skaidrių ... apačioje yra pavyzdys ...>
Kai kurias sąvokas reikės pridėti...

Mums prireiks tam tikrų  sąvokų, žr. pvz \citep{Graf}. 
\begin{definition}[Svorinis grafas]\citep{Graf}
Grafas $G=(V,E):\forall e\in E, \exists w:E\to \mathbb{R}^{+}$ vadinamas svoriniu grafu.
\end{definition}

Labai patartina prieš kiekvieną sąvoką parašyti ,,motyvaciją`` kam ji reikalinga... Tai gerokai pagerina teksto suvokimą ...

\begin{definition}[Jungi komponentė]\citep{Graf}
Grafo $G=(V,E)$ jungia komponente vadinsime grafo $G$ porgrafį $H\subseteq G$ tokį, kad $H$ yra jungus grafas, t.y.
$H=(V_H, E_H): \forall v,u\in V, \exists P(v,u), v\neq u$, kur $P(v,u)$ yra kelias iš viršūnės $v$ į viršūnę $u$.
\end{definition}


\begin{definition}[Minimalus dengiantis medis]\citep{Graf}
Minimalus dengiantis paprastojo svorinio grafo $G=(V_G, E_G)$ medis $MST(G)$ yra toks medis (jungus grafas neturintis ciklų) $T=(V_T, E_T)$, kad $V_T=V_G$ ir $\sum_{e\in E_T}{w(e)}$ yra mažiausia galima tokia suma.
\end{definition}

\begin{definition}[Svoris ir vidutinis svoris]
Grafo $G=(V,E)$ svoriu laikysime visų jo briaunų svorių sumą $\sum_{e\in E}{w(e)}$. To paties grafo vidutiniu briaunos svoriu vadinsime šių svorių vidurkį $\frac{\sum_{e\in E}{w(e)}}{\vert V \vert}$.
\end{definition}
%%%%%%%%%%%%%%%%%%%%%%%%%%%%%%%%%%%%%%%%%%%%%%%%%%%%%%%%%%%%%%%%%%%%%%%%%
\newpage
\section{Algoritmai}
<Galima permesti iš skaidrių bazinį algoritmą>
Algoritmus, kurie padeda atlikti tyrimą, reikėtų suformuluoti papildomai. 
Prieš kiekvieną formalųjį algoritmą taip pat reikėtų 3-4 ,,motyvacinių`` sakinių,
žinomų faktų apie jo efektyvumą ir pan.


\begin{algorithm}[h]
    \caption{CCB-FIND}
    \textbf{Duota:} Paprastasis svorinis grafas $G=(V,E)$\\
    \textbf{Rasti:} Sunkiausią jungią komponentę CCB
    \begin{algorithmic}[1]
        \STATE $max := -1$
        \STATE $CCB := Graph()$
        \STATE $ccs := []$
        \STATE $ccs := CCs(G)$
        \FORALL{$T \in ccs$}
            \STATE $Ti = PRIM(T,V_T[0])$
            \IF{$W(Ti) > max$}
                \STATE $CCB = Ti$
            \ENDIF
        \ENDFOR
        \RETURN $CCB$
    \end{algorithmic}
\end{algorithm}

\begin{algorithm}[h]\citep{slides}
    \caption{CCs(G)}
    \begin{algorithmic}[1]
        \STATE $ccs := []$
        \STATE $visited := []$
        \FORALL{$v\in G$}
            \STATE $visited \leftarrow False$
        \ENDFOR
        \FORALL{$v\in G$}
            \IF{$visited[v]=False$}
                \STATE $cc := []$
                \STATE $ccs \leftarrow DFS(G,cc,v,visited)$
            \ENDIF
        \ENDFOR
        \STATE $CCS := []$
        \FORALL{$c \in ccs$}
            \STATE $CCS\leftarrow Graph(G,ccs)$
        \ENDFOR
        \RETURN $ccs$
    \end{algorithmic}
\end{algorithm}

\begin{algorithm}[h]\citep{GTWA}
    \caption{DFS(G,cc,v,visited)}
    \begin{algorithmic}[1]
        \STATE $visited[v] := True$
        \STATE $cc\leftarrow v$
        \FORALL{$u\in ADJ(v)$}
            \IF{$visited[u]=False$}
                \STATE $cc := DFS(G,cc,u,visited)$
            \ENDIF
        \ENDFOR
        \RETURN $cc$
    \end{algorithmic}
\end{algorithm}    
\clearpage
\newpage
\section{Algoritmų realizacija ir tyrimų eiga}
Aptariamos realizacijos: naudota programavimo kalba, parinktos duomenų struktūros,
grafų klasės individualių grafų generavimas ....

\section{Tyrimų rezultatai ir jų aptarimas}
\newpage 
\section*{Išvados}
\label{sec:isvados}
\addcontentsline{toc}{section}{\nameref{sec:isvados}}
Šiame darbe mes atlikome ... tyrimus. Po jų analizės priėjome tokių išvadų:

\begin{itemize}
\item išvada
\end{itemize}

\newpage 
\bibliographystyle{plain}
\bibliography{reference}
\newpage 
\appendix
\section*{Priedai. Kodas}
\label{sec:priedas}
\addcontentsline{toc}{section}{\nameref{sec:priedas}}
\definecolor{codegreen}{rgb}{0,0.6,0}
\definecolor{codegray}{rgb}{0.5,0.5,0.5}
\definecolor{codepurple}{rgb}{0.58,0,0.82}
\definecolor{backcolour}{rgb}{1,1,1}
\lstdefinestyle{mystyle}{
    backgroundcolor=\color{backcolour},   
    commentstyle=\color{codegreen},
    keywordstyle=\color{magenta},
    numberstyle=\tiny\color{codegray},
    stringstyle=\color{codepurple},
    basicstyle=\footnotesize,
    breakatwhitespace=false,         
    breaklines=true,                 
    captionpos=b,                    
    keepspaces=true,                 
    numbers=left,                    
    numbersep=5pt,                  
    showspaces=false,                
    showstringspaces=false,
    showtabs=false,                  
    tabsize=2
}
 
\lstset{style=mystyle}
main.py
\lstinputlisting[language=Python]{../py/main.py}
show.py
\lstinputlisting[language=Python]{../py/show.py}


\end{document}

